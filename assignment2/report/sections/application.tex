\subsection{eLib}
In this first of two case studies, we consider the eLib benchmark as available from \url{http://www.win.tue.nl/~aserebre/2IS55/2011-2012/eLib.zip}.
The obtained class diagram is shown in Figure~\ref{fig:eLib}.
\begin{figure}[H]
\center
\includegraphics[width=\textwidth]{sections/results/eLib}
\caption{Class diagram obtained by running Graphviz on the dot file outputted by the developed tool.}
\label{fig:eLib}
\end{figure}
The tool performs the task of outputting the file in several seconds. 
This is a reasonable time considering the relatively small size of eLib. 
The obtained diagram does not match the one presented in Figure~3.9 in the book by Tonella and Potrich~\cite{rev-eng} exactly. 
Our tool does not discover the dependency between the \texttt{Journal} class and the \texttt{User} class. 
Another difference is that they do not show the class \texttt{Main} in the class diagram and we do.
The missing of a dependency is a loss of precision that is not desired but which we have not found the reason for in the code.
The visual quality of the rendered image is good, the class diagram is readable and the general structure of eLib can be derived by looking at the diagram. 
We decided to generate only edges with straight edges and 90 degree angles because curved edges tend to give a very messy looking picture.
Also, we intersecting edges can be distinguished more easily when all edges are straight.

\subsection{CyberNeko HTML}
In this second case study, we compare two versions of the CyberNeko HTML Parser. 
We compare the obtained class diagrams of versions 0.9.5 (Figure~\ref{fig:neko0.9}) and 1.9.14 (Figure~\ref{fig:neko1.9}) and identify similarities and differences between the two versions and study the evolution of the project.
Finally, we try to relate observed changes in the diagrams to discussions on nekohtml-user and nekohtml-developer mailing lists to determine whether a change was explicitly required or rather a byproduct of other work.

To get CyberNeko v1.9.14 to compile, we had to change a few lines.
As documented in classes \texttt{XercesBridge\_2\_0} and \texttt{XercesBridge\_2\_1}, these Xerces bridge classes will not compile with recent versions of Xerces.
Since we were using a recent version of Xerces, we had to make a few modifications to get CyberNeko to compile.
Specifically, we commented out lines 45, 51 and 57 of \texttt{XercesBridge\_2\_0.java} and line 53 of \texttt{XercesBridge\_2\_1}.
We are aware that our modifications might alter the object flow graph, but in our opinion these classes don not represent core functionality and since our only option was to disregard them we think this approach is justified.

\begin{figure}[H]
\center
\includegraphics[width=\textwidth]{sections/results/{{neko-0.9}}}
\caption{Class diagram obtained by running Graphviz on the dot file outputted by the developed tool applied on CyberNeko version 0.9.5.}
\label{fig:neko0.9}
\end{figure}


\begin{figure}[H]
\center
\includegraphics[width=\textwidth]{sections/results/{{neko-1.9}}}
\caption{Class diagram obtained by running Graphviz on the dot file outputted by the developed tool applied on CyberNeko version 1.9.14.}
\label{fig:neko1.9}
\end{figure}

In a system as large as CyberNeko, it is of very little use to consider all small changes.
Therefore we identify three classes that seem the most important in CyberNeko version 0.9.5 and focus on them.
These are: \texttt{HTMLScanner}, \texttt{HTMLTagBalancer} and \texttt{ParserConfigurationSettings}.
In v1.9.14, the \texttt{HTMLScanner} class is associated with class \texttt{FilterInputStream}, this association was not present in v0.9.5. It was then associated with \texttt{PlaybackInputStream}, which extended \texttt{FilterInputStream}
The \texttt{HTMLTagBalancer} class is in v1.9.14 associated with class \texttt{ElementEntr}y, this association was not present in v0.9.5.
The class \texttt{ParserConfigurationSettings} stands out in the class diagram of v0.9.5 as three different classes depend on it. In v1.9.14, this has reduced to one.

The classes that are not connected to the rest of the classes (in the top right of the diagram), show little change over the versions. The quality of our tool can be doubted as there are some strange constructs apparent in the class diagram. Such as classes that have two dependencies to the same other class, classes that are associated with themselves or even a class that extends itself.

In general, the class diagram for v0.9.5 is a lot wider than that of v1.9.14.
The way we constructed the class diagrams makes that classes with a lot of incoming arcs are lower in the diagram than classes with less incoming arcs.
Classes on the same 'row' have a similar rank.
In v0.9.5 there were about three ranks with a lot of the classes in them while v1.9.14 has a lot more different ranks.
This implies that the classes in v1.9.14 are structured in a more hierarchical way than the classes in v0.9.5, which are structured in a way more loose, flat way.
Just going from this, we note that the structure of v1.9.14 is a lot better than that of v0.9.5.

An interesting thing to notice in regard to the \texttt{nekohtml-developer} mailing list (which is used primarily to share patches and bug reports) is that it became active for the first time in December of 2007, when v0.9.5 was first released.
This level of activity was more of less maintained throughout 2008 and 2009.
After February of 2010, when v1.9.14 was released, the level of activity abruptly went down.
Since then, the activity level has never again been as high as between v0.9.5 and v1.9.15.
We expected that this might have been due to a practical reason, for instance the discussion moving to another platform, but we were not able to find any evidence of this, so we assume the project just became much less active after the release of v1.9.14.

The \texttt{nekohtml-user} mailing list (which is used for announcements and sometimes help) is much less consistent and it's hard to see any real patterns there.
The interesting part is that one of the few times that this list have been relatively active was in May of 2010, just after the developer mailing list became much less active.
