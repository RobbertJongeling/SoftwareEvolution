\documentclass[a4paper,twoside, twocolumn, 11pt]{article}
\usepackage{a4wide,graphicx,subfigure,fancyhdr,amsmath,amssymb,enumerate,hyperref, float,appendix}
\usepackage[english]{babel}
\numberwithin{equation}{section}

%----------------------- Macros and Definitions --------------------------

\setlength\headheight{20pt}
\addtolength\topmargin{-10pt}
\addtolength\footskip{20pt}

\setlength{\parskip}{4pt}
\setlength{\columnsep}{20pt}

\newcommand{\N}{\mathbb{N}}
\newcommand{\ch}{\mathcal{CH}}

\newcommand{\solution}[1]{\noindent{\bf  #1)}}

\fancypagestyle{plain}{%
\fancyhf{}
\fancyhead[LO,RE]{\sffamily\bfseries\large}
\fancyhead[RO,LE]{\sffamily\bfseries\large }
\fancyfoot[LO,RE]{\sffamily\bfseries\large }
\fancyfoot[RO,LE]{\sffamily\bfseries\thepage}
\renewcommand{\headrulewidth}{0pt}
\renewcommand{\footrulewidth}{0pt}
}

\pagestyle{fancy}
\fancyhf{}
\fancyhead[RO,LE]{\sffamily\bfseries\large Assignment 3 2IS55}
\fancyhead[LO,RE]{\sffamily\bfseries\large Code duplication}
\fancyfoot[LO,RE]{\sffamily\bfseries\large }
\fancyfoot[RO,LE]{\sffamily\bfseries\thepage}
\renewcommand{\headrulewidth}{1pt}
\renewcommand{\footrulewidth}{0pt}


\newcounter{reqc}
\setcounter{reqc}{0}
\providecommand{\req}[1][req \arabic{reqc}]{
	\refstepcounter{reqc}%
	\label{#1}%
	\noindent\textbf{Req.\arabic{reqc}:}
}
%-------------------------------- Title ----------------------------------

\title{\vspace{-\baselineskip}\sffamily\bfseries Assignment 3 2IS55: Code duplication}

\author{Jongeling, R.M. - 0747896 - {\tt r.m.jongeling@student.tue.nl}}

\date{\today}

%--------------------------------- Text ----------------------------------

\begin{document}

\maketitle

\begin{abstract}

\end{abstract}

\section{Introduction}
The purpose of this paper is to perform a replication of the study performed by Livieri et al. in 2007 \cite{paper}.
In this study, they analyze the evolution of the Linux Kernel using code-clone coverage.
In this paper, we follow their approach as closely as possible to analyze the evolution of Azureus/Vuze.
%//TODO add content

\section{Workflow}
In this section, we will discuss the techniques used to perform the study.
The study consists of three clearly separable steps, preprocessing, code-clone finding and post-processing.
The preprocessing step would normally consist of downloading the tool ccFinder and the code of Azureus/Vuze.
As we use the provided Virtual Machine to perform this assignment, the preprocessing step had already been performed.

\subsection{Clone detection}
In the code-cloning step, we chose to write a Bash script performing all comparisons needed.
The script we wrote is attached in Appendix \ref{script}.
To run the script, place it within a subfolder of the \texttt{ccfinder-src/ubuntu32} folder on the VM. 
Then browse to that folder using the command line and run the script by running the command ./script (where``script") is the name of the script.
If this does not work, make sure the checkbox``Allow executing file as program" is checked in the properties of the file (RMB on file, select properties, then choose tab ``Permissions")

%If you decided to develop some of your tools yourself (e.g., scripts) please describe the functionality of these tools and add a listing as an appendix.
In the script, we first list all folders in the \texttt{/home/ubuntu/ccfinder-src/source/source} folder.
Each of these folders contains a single version of the Azureus/Vuze program.
Then, we loop over all folders and determine the number of lines of code of each version.
This is done in the same way as Livieri et al. describe, using the command \texttt{wc -l}.
In the inner loop, we again loop over all folders, such that we can do a pairwise comparison of the versions.
We make sure to only compare the versions one-way, the clones between version $x$ and version $y$ are of course exactly the same as between version $y$ and version $x$.
Also, we do not compare versions to themselves as the value of $Coverage_{ii}(i,i)$ would trivially yield one.

The most important command in the script is command \texttt{ccfx d java -dn <v1> -is -dn <v2> -w f-w-g+}. 
Where $<v1>$ and $<v2>$ correspond to a path to a folder containing a version of the program.
This command detects clones between the two versions, $folder$ and $folder2$ with minimal token length 50, indicated by \texttt{-b 50}.
The \texttt{d java} means, use execution mode \texttt{d}, for detection and consider only Java source files.
The \texttt{-dn} part makes sure we consider all files in the specified folder.
\texttt{-is} is a separator, indicating that the two specified folders belong to different groups. 
This is used in the part \texttt{-w f-w-g+}, which makes sure we do not detect code clones within files or between files of the same file group but only between files from the distinct file groups.

\subsection{Post processing}
The output of the script is for every comparison a line in the following format: $V_i$, $V_j$, $Loc(C_{ij})$,$ Loc(V_i)$, $Loc(V_j)$.
The metric can then be calculated by dividing $Loc(C_{ij})$ by the sum of $ Loc(V_i)$ and $Loc(V_j)$.
%//TODO more


\section{Case study}
In this section, we discuss the code clone detection possibilities of the tool CCFinder as well as the studied program Azureus/Vuze.

\subsection{CCFinder}


\subsection{Azureus/Vuze}
%What is its purpose?
%How old is it?
%How big is it now (number of lines of code, files and packages)?
%Which versions of the application have you considered?
%Why have you considered these versions?
Vuze is a bittorrent client aimed to be ``an end-to-end software application for all your torrent needs" \footnote{vuze.com}.
The project first started under the name ``Azureus" in June 2003, from June 2008, the project has continued under the name ``Vuze".
At the time of this writing, the last stable release is version 5.3.0.0 and dates from February 7, 2014. \footnote{http://sourceforge.net/projects/azureus/}
%//TODO It contains 3.887 files, y lines of code and has a size of 31.5 MB
Azureus/Vuze is one of the most successful open source projects on Sourceforge, it has been downloaded a total of 540.2 million times from Sourceforge \footnote{http://sourceforge.net/top/}.

We have considered all 79 versions of the program of which the source code had been placed on the VM. 
These are interesting versions because it contains early versions, from when it was known as ``Azureus" as well as later versions, when the program was renamed to ``Vuze".
There were more changes to the program than just a name change %//TODO source, uitleg
and it is interesting to investigate if this is visible in the clone ratios. %//TODO clone ratios?

\section{Results and discussion}

\section{Conclusions}

\begin{thebibliography}{9}
\bibitem{paper}
S. Livieri, Y. Higo, M. Matsushita, K. Inoue. \emph{Analysis of the Linux Kernel Evolution Using Code Clone Coverage.} Proceedings of the Fourth International Workshop on Mining Software Repositories (MSR 2007),  pp. 22-1 22-4, Washington, DC, USA, 2007.
\end{thebibliography}

\onecolumn
\pagestyle{empty} %//TODO empty pagestyle or similar to rest of report?
\begin{appendices}
\section{Bash script}\label{script}
\scriptsize
\begin{verbatim}
#!/bin/bash

#list all folders countaining a version of Azureus/Vuze
folders=$(ls /home/ubuntu/ccfinder-src/source/source)

#loop doing all pairwise comparisons and calculating the metric
for folder in $folders; do
    #determine sloc, as per the paper, with using wc -l.
    declare -i sloc_$folder
    sloc[$folder]=$(( find /home/ubuntu/ccfinder-src/source/source/$folder 
        -name '*.java' -print0 | xargs -0 cat) | wc -l)

    #do comparissons
    for folder2 in $folders; do    
        #make sure to triangulate, not rectangulate
        if [ $folder -gt $folder2 ]; then     
            ../ccfx d java -b 50 -dn /home/ubuntu/ccfinder-src/source/source/$folder 
                -is -dn /home/ubuntu/ccfinder-src/source/source/$folder2 -w f-w-g+ 
                    -o $folder-$folder2.ccfxd 

            #filter output file for RNR < 0.5
            ../ccfx m $folder-$folder2.ccfxd -c -o clone-$folder-$folder2.tsv            
            ../picosel -o toremain-$folder-$folder2.txt from clone-$folder-$folder2.tsv 
                select CID where RNR .ge. 0.5
            ../ccfx s $folder-$folder2.ccfxd -o $folder-$folder2-filtered.ccfxd -ci 
                toremain-$folder-$folder2.txt

            #calculate number of lines per clone
            ../ccfx m $folder-$folder2-filtered.ccfxd -w 
                -o line-$folder-$folder2.tsv cat line-$folder-$folder2.tsv
               |awk 'NR>1{print $4}' > cloc-$folder-$folder2.txt
            Cij=$(paste -sd+ cloc-$folder-$folder2.txt | bc)

            #output
            echo $folder,$folder2,$Cij,${sloc[$folder]},${sloc[$folder2]} 
                >> output.txt

            #remove all temporary created files now
            rm -f $folder-$folder2.ccfxd
            rm -f clone-$folder-$folder2.tsv
            rm -f toremain-$folder-$folder2.txt 
            rm -f $folder-$folder2-filtered.ccfxd
            rm -f line-$folder-$folder2.tsv
            rm -f cloc-$folder-$folder2.txt
            rm -f clone-$folder-$folder2-filtered.tsv
        fi
    done
done



\end{verbatim}
\end{appendices}



\end{document}
