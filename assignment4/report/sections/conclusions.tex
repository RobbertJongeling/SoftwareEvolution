We conclude that there are not enough projects for which the p-value of the chi-squared test is lower than 0.05 i.e. for which we can refute the null hypothesis.
Therefore we conclude that there is no statistically significant difference in the chance of a build to succeed or fail when a change is introduced via a commit or a pull request.

Other interesting results of the study are differences observed between programming languages. 
We have seen that for Python and Ruby, on average more contributors are involved in a project compared to Java.

Also, we have found that Java projects tend to have relatively more commits and fewer pull requests as compared to Python and Ruby projects.
Thus, it seems like Java projects are less community-oriented than Python and Ruby projects are.
Our intuitive explanation for this fact is that Java projects are usually more enterprise oriented.
However, since we've only analyzed \textbf{public} GitHub repositories and enterprise projects are typically not openly available, more research is needed to determine if this claim is correct.

Furthermore we have seen that relatively old, active projects with relatively few contributors have a larger percentage of projects refuting the null hypothesis.
