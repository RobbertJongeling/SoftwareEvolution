In this section we discuss GitHub and Travis CI.

\subsection{GitHub}
GitHub can be used to host software projects that use the \texttt{git} revision control system. 
The GitHub system provides a number of services related to managing these projects. 
A project is hosted in a repository. 
Several developers can contribute to a repository by means of commits to this single repository.
This is calle the Shared Repository Model.
 
A different way of contributing to a repository is the Fork and Pull model.
In this model, anyone is allowed to fork a repository and make changes to their own fork.
To make the changes to the main repository, a pull request must be issued.
The project maintainer will then accept or deny this request.
On acceptance of the pull request, the changes made in the fork will be conducted in the main repository.

Furthermore, GitHub provides functionality to improve collaboration of multiple developers on the same project.
Issues can be created and assigned to a contributor. 
On every line of every commit any contributor can comment, this enhances the collaborative code review process.

\subsection{Travis CI}
Travis is a system that enables developers to easily set up continuous integration.
Continuous integration is the practice of building a software project from source to product frequently.
Completing the entire build process on a regular basis should prevent integration problems that can occur once the project is due for a release.

Normally, most developers only test their parts of the system when they have made a change.
With continuous integration, the goal is to test the entire system by any availble means.
This includes unit testing, but can also include the generation and deployment of documentation or the deployment of the software itself.
Usually, a log of the test results is kept so progress can be visualized.

Travis is a hosted service, which means users do not have to set up software themselves, but can use the Travis platform.
Furthermore, Travis makes it really easy for GitHub users to use the service, as every public GitHub repository already has a Travis profile.
The only thing repository maintainers have to do is enable a callback to Travis whenever an interesting event happens and configure Travis by checking a configuration file called \texttt{.travis.yml} into the root of their repository.

In the case of Travis, buidling 'frequently' is every time a change is pushed to (any branch of) the repository on GitHub or when someone issues a pull request.
Travis keeps a log of everything that has happened during the build process, enabling developers to inspect what went wrong in case of a failure.

In addition, Travis can be configured to send notifications when a build has completed, for instance through email or IRC.
