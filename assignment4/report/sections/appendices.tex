\begin{appendices}
\section{First query} \label{app:firstquery}
\begin{scriptsize}
\begin{verbatim}
select id, name, language, TIMESTAMPDIFF(day, DATE_FORMAT(projects.created_at,'%Y-%m-%d'), '2014-03-06') as age_in_days 
  from projects where 
    projects.deleted = 0 and
    projects.forked_from is null and
    (projects.language='Ruby' or projects.language='Java' or projects.language='Python') and 
    TIMESTAMPDIFF(year, DATE_FORMAT(projects.created_at,'%Y-%m-%d'), '2014-03-06') >= 1 and 
    (select count(user_id) from project_members where project_members.repo_id=projects.id) >= 10 
    order by projects.id asc
\end{verbatim}
\end{scriptsize}

\section{Small queries} \label{app:smallqueries}
Here we show the smaller queries executed to retrieve more detailed data of the projects. All the project ids of the projects found in the first query are listed and here abbreviated by \texttt{allIDs}.
\begin{scriptsize}
\begin{verbatim}
--get number of contributors per repository
select count(user_id) from project_members where 
  repo_id in (allIDs) GROUP BY repo_id order by repo_id asc
	
--get total nr pull requests per project --not opened because all that matters is check that not 0
select base_repo_id, count(id) from pull_requests where 
  base_repo_id in (allIDs) group by base_repo_id order by base_repo_id asc

--get total nr commits per project
select project_id, count(commits.id) from commits where 
  commits.project_id in (allIDs) group by project_id order by project_id asc

--select owner
SELECT projects.id, users.login FROM projects INNER JOIN users ON projects.owner_id=users.id  where 
  projects.id in (allIDs) 
\end{verbatim}
\end{scriptsize}

\section{Last month queries} \label{app:lastMonth}
Here we show the smaller queries executed to retrieve commit and pull request data in the last month. All the project ids of the projects found in the first query are listed and here abbreviated by \texttt{allIDs}. Here we also check if the time difference is greater than or equal to zero months because the database contains records of commits in the future.
\begin{scriptsize}
\begin{verbatim}
--get number of commits and in last month per project
select project_id as pid, count(id) as cid from commits where 
    project_id in (allIDs) and 
    TIMESTAMPDIFF(month, DATE_FORMAT(commits.created_at,'%Y-%m-%d'), '2014-03-06') <= 1 and 
    TIMESTAMPDIFF(month, DATE_FORMAT(commits.created_at,'%Y-%m-%d'), '2014-03-06') >= 0 group by project_id

--get nr pull requests and age in month per project 
select pull_requests.base_repo_id, count(pull_request_id) from 
  pull_request_history inner join pull_requests on pull_request_id=pull_requests.id and 
    base_repo_id in (8,34) and 
    pull_request_history.action='opened' and 
    TIMESTAMPDIFF(month, DATE_FORMAT(pull_request_history.created_at,'%Y-%m-%d'), '2014-03-06') <= 1 and 
    TIMESTAMPDIFF(month, DATE_FORMAT(pull_request_history.created_at,'%Y-%m-%d'), '2014-03-06') >= 0 
    group by base_repo_id order by base_repo_id
\end{verbatim}
\end{scriptsize}

\section{R-script}\label{app:R-script}
\begin{scriptsize}
\begin{verbatim}
#load lib
library("epitools")

#read input
dataFrame = read.csv(file=<path to input file>, header=TRUE,sep=',')

#function doing all the work
f <- function(r) {
    
    if(r["relevant"] == "true") {    
        #create contingency table, map strings as read from csv to numeric
        x <- c(r["commit_passed"], r["pr_passed"], r["commit_failed"], r["pr_failed"])
        x <- as.numeric(x)
        ctable <- matrix(x, ncol = 2)
        
        #perform chi test on contingency table
        chitest <- chisq.test(ctable)
        pofchisq <- chitest$p.value
        
        #calculate residuals, odds ratio and its p-value
        res <- chitest$residuals
        res11 <- res[1,1]
        res12 <- res[1,2]
        res21 <- res[2,1]
        res22 <- res[2,2]
        or <- oddsratio(ctable)
        oddsRatio <- or$measure[2,1]
        pValue <- or$p.value[2,1]    
    } else {
        pofchisq <- "NA"
        res11 <- "NA"
        res12 <- "NA"
        res21 <- "NA"
        res22 <- "NA"
        oddsRatio <- "NA"
        pValue <- "NA"
    }
    #perform output
    #print(paste(r["ghtorrent_id"], pofchisq, res11, res12, res21, res22, oddsRatio, pValue, sep=","))
    cat(paste(r["ghtorrent_id"], pofchisq, res11, res12, res21, res22, oddsRatio, pValue, sep=","), 
              file=<path to outputfile>, append = T, fill = T)
}

#applying the function to the input
apply(dataFrame, 1, f)

\end{verbatim}
\end{scriptsize}
\end{appendices}
