In this report we study a number of large, active software projects using GitHub and Travis, a repository hosting service and a continuous integration (CI) tool respectively.
Travis CI automatically builds a project when a change is made, such a change can be made by either a commit or a pull request.
The purpose of this study is to determine what, if any, differences of the probability of a build to fail there are between changes introduced by commits and changes introduced by pull requests.

To this end, we consider large and active independent projects hosted on GitHub that use Travis CI.
Large and active independent projects are defined to be projects that:
\begin{enumerate}
\item are not forks of other repositories,
\item have not been deleted,
\item have at least 10 different contributors,
\item have at least 10 changes (commits or pull requests) during the previous month,
\item are at least one year old,
\item have both commits and pull requests,
\item have been developed in Java, Python or Ruby.
\end{enumerate}
Of these projects, we are interested in analyzing the ones that use Travis CI.

%In Section \ref{background}, we briefly discuss GitHub and Travis CI.
%Next, in Section \ref{methodology}, we describe how we obtained the data and what analysis we have performed on the obtained data.
%In Section \ref{results}, we present the obtained data and results of the statistical analysis.
%Then, we briefly reflect on threats to the validity of our results in Section \ref{threats}.
