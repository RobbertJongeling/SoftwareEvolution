In this section we present and analyze the results obtained from the various steps as described in Section~\ref{methodology}.

\subsection{Identified repositories}
We identified 734 projects adhering to our definition of large and active independent projects.
On average these projects have 49.4 contributors. 
Note that these are all contributors with commit access to the repository.
The project with the most amount of contributors (440) also has the most changes in the last month (4441).
The average amount of changes in the last month is 141.9.
The average age of the identified projects is 966.0 days. 
As we are looking for large projects, that are also at least one year old, this average is likely much higher than the overall average age of all GitHub projects.

Of the total of 734 identified projects, 263 are written in Java, 239 in Ruby and 232 in Python. 
This is a bit surprising because in 2012, there were almost twice as many Ruby projects than Python projects created on GitHub. \footnote{http://adambard.com/blog/top-github-languages-for-2013-so-far/}

In Table~\ref{tab:allChanges} we present the average number of commits and pull requests per programming language.
\begin{table}[h]
\begin{tabular}{ l | l l }
 & commits & pull requests\\
\hline
Java & 2371.9 & 197.8 \\
Python & 1833.6 & 332.1 \\
Ruby & 1893.3 & 341.2
\end{tabular}
\caption{Average number of commits and pull requests of the identified projects per programming language.}
\label{tab:allChanges}
\end{table}
A clear difference can be seen between Java and the other programming languages. 
Where the total number of changes is similar across programming languages, in Java a third more changes are made through commits as compared to Python and Ruby.
This might have to do with the difference in the number of contributors to projects, an active community where more people are working together should yield more pull requests.
We see that the average number of contributors for the identified Java projects is 43.4, whereas this average is 53.4 and 52.2 for Python and Ruby projects respectively.

\subsection{Travis}
Of the identified 735 projects, there are 309 projects for which at least one of the fields \textit{commits passed, commits failed, pull requests passed} and \textit{pull requests failed} is greater than zero. 
Of these projects, 171 are written in Ruby, 96 in Python and 42 in Java. 
When we compare these numbers to the total number of projects identified we see that 71.5\% of the Ruby projects use Travis CI.
This is considerably more than Python (41.4\%) and Java(15.9\%).
We have not found a satisfying explanation for this large difference.

When considering the projects that do use Travis, we find that on average, there are 2428 changes made in total. 
Of these changes, 367 (15.1\%) are pull requests.
For the projects that do not use Travis, this is 221 pull requests on 2200 changes (10\%).


\subsubsection*{Not enough data}
As mentioned, there have been 309 projects identified to use Travis CI.
Of these 309, 193 projects had a value of 5 or higher in each of those fields, which makes them suitable for out statistical analysis.
This means that 114 projects that did use Travis CI do not have enough data to be analyzed. 
Of these projects, the fields concerning the number of commits have a value less than 5 more often than the fields concerning pull requests.

The field that most often contains a value of less than 5 is the field concerning failed builds after a commit, in 90 of the 114 cases, this value is less than 5.
This can be explained by the fact that this is also the field that has the lowest average value for all projects and these four fields. 
The average value of the fields \textit{commits passed, commits failed, pull requests passed} and \textit{pull requests failed} is 231.7, 90.6, 350.2 and 147.6 respectively.
The fraction of failed commits of the total number of commits is 28.1\%. This is very similar to the fraction of failed pull requests of the total number of pull requests, which is 29.7\%.
The fact that the number of commits is lower than the number of pull requests for these 309 projects may have to do with the use of Travis CI, which indicates a more professional approach to the development which would yield more people to work in the fork and pull model.

Of the projects that 114 projects that do not have enough data to be suitable for our statistical analysis, 55 are written in Ruby, 37 in Python and 22 in Java.
This is an interesting observation as we saw earlier that the number projects is evenly distributed over the three programming languages.
There is no clear reason for this difference between the programming languages.

\subsection{Statistical analysis}
Now, we discuss the results of the statistical analysis on the 193 projects that have enough data to perform the statistical analysis as described in Section~\ref{methodology}.
Of these 193 projects, 84 (43.5\%) have a p-value of the chi-squared test of less than 0.05. 
We can not refute the null hypothesis that the success of a build is independent of the way a code change is introduced.
With confidence intervals of 0.01 and 0.1 there are 68(35.2\%) and 96 (49.7\%) projects for which the null hypothesis can be refuted respectively.

When comparing different programming languages we observe few differences between the languages.
Table~\ref{tab:p-for-lang} contains for each of the studied languages the number of projects having a p-value lower than 0.05.
\begin{table}[h]
\begin{tabular}{ l | l l l}
 & p $<$ 0.05 & total & \%\\
\hline
Java & 7 & 19 & 36.8\\
Python & 31 & 59 & 52.5 \\
Ruby & 46 & 115 & 40.0
\end{tabular}
\caption{Number of projects per programming language for which the p-value is smaller than 0.05.}
\label{tab:p-for-lang}
\end{table}
There is a relatively large difference between Python and the other languages. 
For the Java projects there are too few projects to make meaningful observations on this result.

When comparing the projects on age we identify three categories; age between 365 and 730 days, age between 731 and 1461 days and older projects.
Table~\ref{tab:p-for-age} contains for each of the studied ages the number of projects having a p-value lower than 0.05.
\begin{table}[h]
\begin{tabular}{ l | l l l}
 & p $<$ 0.05 & total & \%\\
\hline
1-2 years & 15 & 45 & 33.3\\
3-4 years & 49 & 108 & 45.4 \\
5+ years & 20 & 40 & 50.0
\end{tabular}
\caption{Number of projects per age group for which the p-value is smaller than 0.05.}
\label{tab:p-for-age}
\end{table}
We see that for older projects there are more rejections of the null hypothesis. 

When comparing the projects on the number of contributors we identify three categories; between 10-24 contributors, 25-50 contributors and 51 or more contributors.
Table~\ref{tab:p-for-contributors} contains for each of the studied number of contributors the number of projects having a p-value lower than 0.05.
\begin{table}[h]
\begin{tabular}{ l | l l l}
 & p $<$ 0.05 & total & \%\\
\hline
10-24 contributors & 54 & 104 & 51.9\\
25-50 contributors & 20 & 47 & 42.6 \\
50+ contributors & 10 & 42 & 23.9
\end{tabular}
\caption{Number of projects per grouped number of contributors for which the p-value is smaller than 0.05.}
\label{tab:p-for-contributors}
\end{table}
We see that for projects with more contributors there are fewer where there is a statistically significant relation between the success of a build and the way the changes are introduced.

When comparing the projects on the number of changes in the last month, we identify three categories; between 10-49 changes, 50-140 changes and more than 150 changes.
Table~\ref{tab:p-for-changes} contains for each of the studied number of changes in the last month the number of projects having a p-value lower than 0.05.
\begin{table}[h]
\begin{tabular}{ l | l l l}
 & p $<$ 0.05 & total & \%\\
\hline
10-49 changes & 22 & 68 & 32.4\\
50-150 changes & 22 & 61 & 36.1 \\
150+ changes & 40 & 64 & 62.5
\end{tabular}
\caption{Number of projects per grouped number of changes in the last month for which the p-value is smaller than 0.05.}
\label{tab:p-for-changes}
\end{table}
An interesting observation here is that from the active projects, a lot more refute the null hypothesis. 
This indicates that for more active projects, there is more likely a relation between the success of a build and the way the changes are introduced.





















