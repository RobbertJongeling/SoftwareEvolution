In this section we discuss the steps we perform to collect and analyze GitHub repositories.

\subsection{Finding projects on GHTorrent}
In this research, we collect meta-data of GitHub repositories and use this meta-data to select large and active independent projects.
This meta-data can be queried using the relational database GHTorrent. \cite{ghtorrent}
We query the GHTorrent MySQL database by using the web interface available at \url{ghtorrent.org/dblite}. 
Because of the instability of this endpoint, we decided to use relatively simple queries and do most of the filtering of the resulting csv files in a spreadsheet program.

First, we queried for repositories adhering to points 1,2,3,5 and 7 as enumerated in the definition of large and active independent projects.
The query used is included in appendix \ref{app:firstquery}.
The database contains data up to and including march 6th 2014 so in this and following queries we chose that date to relate the age of projects, commits and pull requests to.

We then execute a number of small queries to gather information such as the number of contributors per repository. 
We also queried the projects owner and the total number of commits and pull requests per project. 
From them, we get information that enables us to filter on point 6 of the definition of large and active independent projects. 
This information also enables us to provide the necessary input for the Travis API.

These queries have been included in appendix \ref{app:smallqueries}. 
They do not necessarily return a value for all project ids found in the first query.
In order to get a complete list, we add values of 0 for every project for which no total number of commits or total number of pull requests has been found.

We now only need to verify point 4 of the definition of large and active independent projects. 
The way of working is very similar to the execution of the smaller queries. 
The queries to retrieve the number of commits and the number of pull requests in the last month per project are included in appendix \ref{app:lastMonth}

\subsection{Retrieving information from the Travis CI API}

\subsection{Statistical analysis with R}
For each large and active independent project that uses Travis CI, we have now identified four numbers; for both commits and pull requests the number of corresponding builds to pass and fail.
For each projects, we added a column \texttt{relevant} which is "true" if and only if all these numbers are greater than or equal to 5.
We perform chi-squared test on the input and as known in the literature, a good rule of thumb for the chi-squared test is that the value of each cell in the contingency table is at least 5.
Therefore we are not interested in the other projects and the value of the column \texttt{relevant} for those projects is "false".

As mentioned, we perform a chi-squared test on the contingency table of every project.
This contingency table is a 2x2 matrix with rows corresponding to commits and pull requests and columns corresponding to passed and failed builds.
After performing the chi-squared test, we calculate its p-value and residuals.
Then, we calculate the odds ratio of the contingency table and report its measure and p-value. 
The script used to perform the statistical analysis has been included in Appendix \ref{app:R-script}

%//TODO wellicht een bronvermelding van de opdracht tekst, dit is er wel erg op geinspireerd
We perform the chi-squared test in order to find out whether the values in the contingency table are independent or not, i.e., whether the success or failure of a build is independent from the way the change is made.
The null hypothesis is that the values are independent, to judge whether the result is statistically significant, we calculate the p-value.
We will state that the null hypothesis can be rejected if the p-value is lower than 0.05.

We inspect the residuals of the chi-squared test to find out how the number of occurrences %//TODO of what?
observed relates to the expected number under the independence assumption.
To formalize the strength of this dependence, we calculate the odds ratio and corresponding p-value. 
Here, we also use the standard 95\% confidence interval

